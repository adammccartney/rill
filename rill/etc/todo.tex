% Created 2020-11-02 Mon 15:35
% Intended LaTeX compiler: pdflatex
\documentclass[11pt]{article}
\usepackage[utf8]{inputenc}
\usepackage[T1]{fontenc}
\usepackage{graphicx}
\usepackage{grffile}
\usepackage{longtable}
\usepackage{wrapfig}
\usepackage{rotating}
\usepackage[normalem]{ulem}
\usepackage{amsmath}
\usepackage{textcomp}
\usepackage{amssymb}
\usepackage{capt-of}
\usepackage{hyperref}
\author{adam}
\date{\today}
\title{Define persistent data structure for material in segments}
\hypersetup{
 pdfauthor={adam},
 pdftitle={Define persistent data structure for material in segments},
 pdfkeywords={},
 pdfsubject={},
 pdfcreator={Emacs 26.3 (Org mode 9.1.9)}, 
 pdflang={English}}
\begin{document}

\maketitle
\begin{itemize}
\item worked through a couple of iterations that tried to use a NamedTuple to manage the data used to initialize the music functions as they are passed into the SegmentMaker member functions before being typeset as notation. Problem is that it is seemingly not so easy to use user defined and complex types in a namedtuple

\item use of python's DataClass seems to facilitate working with (more) complex datatypes and custom classes

\begin{itemize}
\item I'm trying to get around having to hand-hack the initialization of the MusicMaker objects: so I want to be able to generate the "definition.py" scripts on a per segment basis, and not have to tweak the output. So ideally, each "definition.py" would refer to an "init segment music" data class on a per instrument basis in each segment. This means that we could write a generator script for all instances of "definition.py" within the subdirectories of segments
\end{itemize}
\end{itemize}
\end{document}
