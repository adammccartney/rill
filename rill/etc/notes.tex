% Created 2020-10-27 Tue 19:42
% Intended LaTeX compiler: pdflatex
\documentclass[11pt]{article}
\usepackage[utf8]{inputenc}
\usepackage[T1]{fontenc}
\usepackage{graphicx}
\usepackage{grffile}
\usepackage{longtable}
\usepackage{wrapfig}
\usepackage{rotating}
\usepackage[normalem]{ulem}
\usepackage{amsmath}
\usepackage{textcomp}
\usepackage{amssymb}
\usepackage{capt-of}
\usepackage{hyperref}
\author{adam}
\date{\today}
\title{}
\hypersetup{
 pdfauthor={adam},
 pdftitle={},
 pdfkeywords={},
 pdfsubject={},
 pdfcreator={Emacs 26.3 (Org mode 9.3.8)}, 
 pdflang={English}}
\begin{document}

\tableofcontents

\section{rill (erosion)}
\label{sec:orgcbe8723}

\subsection{Compositional Idea in v1.0}
\label{sec:org0452364}
a rill is the soft impression left in the soil in the wake of a soft, lightly 
flowing stream of water. in nature, a rill will form part of a network. 

In musical terms, the basic idea will be to map the two elements acoustically: 

\begin{itemize}
\item chaotic (as in mathematics) melodies in poly modal space
\item a delayed tremolo texture that follows the same outline
\end{itemize}

The score is a sequence of short pieces or fragments that explore this
idea.  


\section{Material Definitions}
\label{sec:org0240183}
\subsection{Pitch Material types v1.0}
\label{sec:orgf0c49f8}
\begin{itemize}
\item dictionaries of `abjad.PitchSegment` that convey the information about a set of (tonal/harmonic)
changes
\item lookup tables that are designed to process transformation of
material (mostly transpositions and offsets)
\item natural harmonic tremolo figures occuring on the violin
\item legato tremolo figures that occur
\end{itemize}


\section{Use of Object Oriented Programming}
\label{sec:orgd497e0c}
Generally speaking, the piece attempts to keep the inheritance
hierarchies as flat as possible. Also, it tries to limit the
complexity and size of any single class definitions, preferring
instead to a allow complexity to arise from the combination or
succession of simple classes in a sort of "pipeline" style program. 

As a library, Abjad has some fairly robust class definitions that
make it easy to interact with the fundamental musical types as they
are found in lilypond. 

Lilypond as a language is also in a fairly mature stage of
development, and the basic syntax and usage is well defined and
documented. 

While there is a bit of a learning curve associated with both the
lilypond and (moreso) python languages, as well as the abjad API, the
payoff as a composer is such that once a fundamental understanding has
been attained, one is free to start thinking about music composition
in the type of high-level abstraction that computer programming
languages are really good at handling.  

The basic approach that is being adopted for the context of this score
is to try and think about it as a sort of "interface library" that
interacts with abjad. The reason for doing this is with a view to
retaining as much "orthogonality" as possible within the context of
the design (in this case the composition). 

\subsection{Diad}
\label{sec:orgd47b653}
\subsubsection{v1.0}
\label{sec:org17e4638}
Used primarily to greate tremolo figures on the strings, with this in
mind the pitch material is always a pair of natural harmonics that are
found on the same string. 

\subsection{FuzzyHarmony}
\label{sec:org15491f5}
Stores a collection of notes in a PitchSegment Fuzzy because the names
are used to roughly track the position in a given harmonic sequence
and not to store precise information on the harmony e.g.: bf\textsubscript{ii} = key
of bflat, minor harmony on ii degree (results in a cmin chord)

\subsubsection{v1.0}
\label{sec:orgd38aa9d}
The instances used are usually tetradic (contain 4 pitches)
and use an inversion/figured bass position that obscures the root
triad by utilizing suspended tones.  

\subsection{LegatoArpeggio}
\label{sec:orgf607325}
Basically creates a string of pitches that is used when generating the
pitch aspects of a leaf.

\subsubsection{v1.0}
\label{sec:org42b2830}
As the objects are used in this version, they represent a sort of
"floating" harmonic figure. A chord is essentially augmented in time. 

\subsection{MusicMaker}
\label{sec:orgc6754d8}
The purpose of the music maker class is to prepare a container that
can be passed into the `.notes` member of a `SegmentMaker`
instance. Using this sort of a wrapper makes it slightly easier to
focus on composing with "phrases" as opposed to always referencing
"atomic" note values and passing these to a leaf maker when the
`SegmentMaker` runs. 

\subsection{MusicStream}
\label{sec:orga129853}


\section{General Methodology (point to point procedure)}
\label{sec:org9650dfa}

The general approach to getting music into the score is through the
instantiation of a SegmentMaker class. Basically, this class contains
a ScoreTemplate as a member, which in turn contains the reference information
about "voices" and "staves" contained within the score. 

The score is divided into a number of discrete segments and these are
created as instances of the class in a `semgent\_\$x/definition.py`
file. There is one such file per section of the score. 

\subsection{Approach in v0.1}
\label{sec:org56976f7}
Very simply generating strings of lilypond syntax using vanilla
python, more or less by using simple format strings. All the control
of structures were done manually.   

\subsection{Approach in v1.0}
\label{sec:org32777ea}
Using abjad as a layer to interface with lilypond. The complexity of
the API made it somewhat tricky to understand what was
happening. Basically ended up reading a number of existing source code
repos of other composers (Trevor Baca, Josiah Wold Oberholzer, Gregory
Evans) to see what sort of approaches were being employed. 

Eventually ended up with an approach where the "notes", "dynamics" and
"markup" were prepared in lists and passed into the appropriate
members of the `SegmentMaker` instance. While this enabled some higher
level thinking and use of variables to reference the contents of
custom objects, it still meant practically dealing with all of the
note material on an atomic level.

\subsubsection{Problems}
\label{sec:orge3632f4}
\begin{itemize}
\item no bar checks
\item re-use of variablec
\end{itemize}

\subsection{Approach in v2.0}
\label{sec:orgae85e7b}




\section{Instrumentation}
\label{sec:orgd1a0394}

v0.1
\begin{itemize}
\item flute(s)
\item guitar
\item clarinet Bb
\item viola
\end{itemize}

v1.0
\begin{itemize}
\item Violin
\item Monosynth
\item Polysynth
\end{itemize}

v2.0
\begin{itemize}
\item 4 Flutes
\item 1 Bb Clarinet
\item Percussion
\item 8 Violins
\item 1 Viola
\end{itemize}
\end{document}
