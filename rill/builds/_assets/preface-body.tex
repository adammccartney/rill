\begin{center}
\huge INS WASSER EINGESCHRIEBEN
\end{center}

\begin{center}
\textit{Verfliesset vielgeliebte lieder,\\
Zum Meere der Vergessenheit!\\
Kein Knabe sing' entzückt euch wieder,\\
Kein Mädchen in der Bluthenzeit.}

\textit{Ihr sanget nur von meiner Lieben;\\
Nun spricht sie meiner Treue Hohn.\\
Ihr wart ins Wasser eingeschrieben,\\
So fliesst denn auch mit ihm davon.}

\hspace {4cm} --- Goethe
\end{center}

\textbf{Staves.} The piece is conceived according to an independent interplay
of tempo, pitch, string and bridge contact points, bow speed \& dynamics. Each
part nonetheless notates on a single staff because these parameters change so
slowly with respect to each other.

\textbf{String \& bridge contact points.} Different grades of tasto, ordinario,
ponticello and on-the-bridge bowing appear in the score:

\begin{tabular}{l l l}
\phantom{M} & T & \textit{tasto} \\
            & pT & \textit{poco tasto} \\
            & PO & \textit{pos. ordinario} \\
            & pP & \textit{poco ponticello} \\
            & P & \textit{ponticello} \\
            & MP & \textit{molto ponticello} \\
            & XP & \textit{ponticello possibile} \\
            & OB & directly on bridge \\
\end{tabular}

Fractional bridge contact points like $\frac{1}{3}$OB, $\frac{1}{2}$OB and
$\frac{2}{3}$OB also appear. These are to be interpreted such that the bow
comes into increasingly greater contact with the bridge and increasingly less
contact with the the string. The indications progressively reduce pitch content
while whitening timbre: unqualified OB specifies pure white noise. Play these
with the bow held diagonally. Note that bow contact points (indicating, for
example, the point or talon) are not specified.
 
\textbf{Bow speed.} The score contrasts  widely different speeds of the bow:
 
\begin{tabular}{l l l}
\phantom{M} & XFB & extremely fast bow (always flautando) \\
            & FB & fast bow (always flautando) \\
            & NBS & normal bow speed \\
            & SB & slow bow \\
\end{tabular}

Note that indications of flautando always refer to bow speed. And note, too,
that a crucial effect of the piece depends on the ways in which very slowly
moving changes in pitch are activated by the constant, irregular and quick
motions of the right hand required to execute the fast (FB) and extremely fast
(XFB) types of bow listed here. Do not substitute tasto for the degrees of bow
speed flautando requested in the score: bow speeds combine freely with the
string and bridge contact points given above. Indications for scratch and
individuated clicks of the bow also appear. The first of these results from a
very slow bow and the second from an almost impossibly slow bow.

\textbf{Tempo \& durations.} The piece comprises a first tempo series (quarter
equals 18, 36 or 72 MM) and a second tempo series (quarter equals 27, 54 or 108
MM). Values of the second series stand 3:2 in relation to values of the first.
Double whole-notes (with vertical bars) and quadruple whole-notes (with
vertical bars and stems) appear throughout the piece. Dotted versions of these
notes (worth, respectively, three and six whole-notes) also appear. The
glissandi between the quadruple whole-notes at the beginning of the piece
connect stems rather than note heads but are to be played just like the
glissandi between other notes.

\textbf{Miscellaneous.} The strings of each instrument are tuned as per usual;
no scordatura.
